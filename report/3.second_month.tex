\section{The Belkin's trinity}
\subsection{Towards a theoretical foundation of Laplacian-based manifold methods}

In this paper they present two results: a pointwise probabilistic convergence of \textbf{the extension of the graph Laplacian} $\hat L_n^t$ 

$$C\frac{(4\pi t_n)^{-\frac{k+2}{2}}}{n}\hat L_n^t f(\bf x) \xrightarrow{n\to\infty}\triangle_\mathcal M f(\bf x)$$

and a uniform (pointwise convergence of the operator) one

$$\sup_{x\in\mathcal M, f\in \mathcal F_C}\left| C\frac{(4\pi t_n)^{-\frac{k+2}{2}}}{n}\hat L_n^t f(\bf x) - \triangle_\mathcal M f(\bf x) \right|\xrightarrow{n\to\infty}0$$

here nothing is said about convergence of the spectra, plus there's nothing written about the relationship between $\mathbf{Eig}\hat L_n^t$ (the eigenfunctions and eigenvalues of the extension of the graph Laplacian) and $\mathbf {Eig} \mathbf {L}_n^t$ (the eigenvectors and eigenvalues of the matrix Laplacian).

Theorem 1 is proven by simple analysis arguments and by Hoeffding's formula (probability). Theorem 2 is proven with arguments of Functional Analysis: Ascoli-Arzelà, compact convergence in Sobolev spaces, etc.etc... However, the proof is far from being clear! 

\subsection{Consistency of Spectral Clustering}
 $$ \mathbf{Eig} \mathbf{L}^t_n \xrightarrow[a.s.]{n\to\infty} \mathbf{Eig} L^t $$
It aims at proving the convergence of eigenvalues and eigenvectors of random graph Laplacian matrices for growing sample size. They present two results: one for the normalized Laplacian and one for the unnormalized Laplacian. Since the matrix eigenvectors grow in dimension as the sample size increases, standard convergence arguments can not be applied. However, they show that there exists a function $f\in C(\mathcal M)$ such that the difference between the eigenvector $v_n$ and the restriction of $f$ to the sample converges to $0$

$$||v_n-\rho_nf||\rightarrow 0$$

To do so they see the eigenvector $v_n$ as the restriction to the sample of a continuous eigenfunction $f_n$ of some continuous operator $U'_n$ that acts on the space $C(\mathcal M)$. Then they use the fact that 

$$||v_n-\rho_nf||_\infty = ||\rho_nf_n-\rho_nf||_\infty\leq ||f_n-f||_\infty$$


So, it will just be necessary to show that  $$||f_n-f||_\infty\rightarrow 0$$
 \textbf{compact convergence} of both matrices towards $L^t$ (where $\mathcal M$ is a compact metric space) in the Banach space of the continuous functions $(C, ||\cdot||_\infty)$. 

Compact convergence ensures convergence of spectral properties in the following sense: for isolated eigenvalues of the limit operator $\triangle_\mathcal M$ with finite multiplicity we have convergence of eigenvalues and eigenspaces but not convergence of eigenfunctions. for isolated simple eigenvalues of the limit operator we have also convergence of the eigenfunctions!

The proof consists in three steps:
\paragraph{Step 1} Construct a bounded operator $U'_n$ on the Banach space $(C(\mathcal M), ||\cdot||_\infty)$ such that restricted on the sampled values behaves like $\mathbf{L}'_n$. Then, construct an operator $U'$ such that for the law of large numbers for a fixed $f$ and for a fixed $x$, $U'_nf(x) \xrightarrow U'f(x) $
\paragraph{Step 2} Here they establishes the connection between the spectra of $L'_n$ and $U'_n$. In particular, they prove a one-to-one correspondence between the eigenfunctions and eigenvalues of $U'_n$ and the eigenvectors and eigenvalues of $L'_n$, provided that satisfy $\lambda\notin \{1\}=\sigma_{ess}(U'_n)= \sigma_{ess}(U')$.
\paragraph{Step 3} Here we prove compact convergence:

$$U'_n \xrightarrow[n\to\infty]{c,\ a.s.}U'$$

At the light of compact convergence and on the analysis of the essential spectrum of $U'_n, U'$ and the one-to-one correspondence of the spectra done in step 2, given Proposition 6 \textit{Perturbation results for compact convergence} we get to the following result

\begin{theorem}
	Let $\lambda\neq 1 $ be an eigenvalue of $U'$ and $M\subset \mathbb C$ an open neighborhood of $\lambda$ such that $\sigma(U')\cap M=\{\lambda\}$. Then:
\begin{enumerate}
	\item Convergence of eigenvalues: The eigenvalues in$\sigma(L'_n)\cap M$ converge to $\lambda$ in the sense that every sequence $(\lambda_n)_{n\in\mathbb N}$ with $\lambda_n\in\sigma(L'_n)\cap M$ satisfies $\lambda_n\rightarrow \lambda$ almost surely.
	\item Convergence of spectral projections: There exists some $N\in\mathbb N$ such that for $n>N$, the sets $\sigma(U'_n)\cap M$ are isolated in $\sigma(U'_n)$. For $n>N$, let $Pr'_n$ be the spectral projection of $U'_n$ corresponding to $\sigma(U'_n)\cap M$, and $Pr$ the spectral projection of $U$ for $\lambda$. Then $Pr'_n\xrightarrow p Pr a.s.$
	\item Convergence of eigenvectors: if $\lambda$ is a single eigenvalue, then the eigenvectors of $L'_n$ converge a.s. up to a change of sign: if $v_n$ is the eigenvector
	of $L'_n$ with eigenvalue $\lambda_n$, $v_{n,i}$ its i-th coordinate, and $f$ the eigenfunction of eigenvalue $\lambda$, then there exists a sequence $(a_n)_{n\in\mathbb N}$ with $a_i \in \{+1,-1\}$ such that $\sup_{i=1,...,n} |a_nv_{n,i} - f(X_i)| \rightarrow 0$ a.s. In particular, for all $b \in\mathbb R$, the sets $\{a_nf_n > b\}$ and $\{f > b\}$ converge, that is, their symmetric difference satisfies $P(\{f > b\}\triangle\{a_nf_n > b\}) \rightarrow 0$.
\end{enumerate}
\end{theorem}

A similar theorem is stated also for non-normalized Laplacian matrix, although the arguments stay the same.
\subsection{Convergence of Laplacian Eigenmaps}
Here all the pieces are put together. 

 $$ \mathbf{Eig} \mathbf{ L}^t_n \xrightarrow[a.s.]{n\to\infty} \mathbf{Eig} L^t \xrightarrow{t\to0} \mathbf{Eig} \triangle_\mathcal M $$
 
 \begin{theorem}
 	Let $\lambda_{n,i}^t$ be the ith eigenvalue of $\hat L_n^t$ and $e^t_{n,i}$ be the corresponding eigenfunction (which for each fixed i will be shown to exist for t sufficiently small). Let $\lambda_{i}$ be the ith eigenvalue of $\triangle_\mathcal M$ and $e_{i}$ be the corresponding eigenfunction. Then there exists a sequence $t_n\rightarrow 0$ such that
 	
 	$$\lim_{n\rightarrow\infty} \lambda_{n,i}^{t_n}=\lambda_i$$
 	$$\lim_{n\rightarrow\infty}||e_{n,i}^{t_n}(x) - e_i(x)||_2 = 0$$
 	
 	where the limits are in probability.
 	
 \end{theorem}

\paragraph{Step 1: Spectral convergence of the empirical approximation $\mathbf{ L}_n^t$ to $L^t$}
Recycle the work of \textbf{Consistency of Spectral Clustering} with the analysis of the essential spectrum of the limit operator $\sigma_{ess}(L^t)$.
\paragraph{Step 2: Spectral convergence of the functional approximation $L^t$ to $\triangle$}
Really hard. Uniform operator convergence does not hold, however spectral convergence is still assured in theorem 4.1 of the paper. This part, although really hard, will stay the same!


\section{Pointwise convergence in the HEALPix case}

Getting inspired from the work flow presented above, we try to replicate the first theorem obtained in \textbf{Towards a theoretical foundation of Laplacian-based manifold methods} but not in the case of random Laplacians, but in the case of deterministic ones where the sampling is given by the HEALPix scheme. This analysis will be a preparatory work to further studies, since as we pointed out so far, form this kind of convergence nothing is said about the convergence of eigenvalues and eigenvectors. However, this study could maybe tell us something about the equivariance of the graph with respect to the action of the rotation group $SO(3)$.

\subsection*{Direct proof of convergence}
\begin{itemize}
	\item We can use the Theorem of porte-manteau to prove that $\sum_{i}f(x_{i})\rightarrow\int f(x)\text{d}x$.
	We need to prove that for any surface, the number of point in the surface will converge toward something that is proportional to the suface.
	\item Let us use a different argument. We know by construction that the
	HEALPix sampling cut the sphere into equal areas. Let us note the
	surface of one level $S_{N_{side}}$, and the maximum distance to
	the center point of one surface\textbf{ $d_{N_{side}}$} . Let us
	write the indicator function of the point ${\bf x}$ that belong to
	the surface $\sigma_{i}$: 
	\[
	\mathbb{I}_{\sigma_{i}}({\bf x})=\begin{cases}
	1 & \text{if }x\in\sigma_{i}\\
	0 & \text{otherwise.}
	\end{cases}
	\]
	Let us assume that the function $f$ is $L$ lipschitz, we have
	\[
	\int_{\sigma_{i}}f({\bf x})\text{d}{\bf x}-\frac{1}{2}Ld_{N_{side}}S_{N_{side}}\leq S_{N_{side}}f({\bf x}_{i})\leq\int_{\sigma_{i}}f({\bf x})\text{d}{\bf x}+\frac{1}{2}Ld_{N_{side}}S_{N_{side}}.
	\]
	Hence we have:
	\[
	\sum_{i}\int_{\sigma_{i}}f({\bf x})\text{d}{\bf x}-\sum_{i}\frac{1}{2}Ld_{N_{side}}S_{N_{side}}\leq S_{N_{side}}\sum_{i}f({\bf x}_{i})\leq\sum_{i}\int_{\sigma_{i}}f({\bf x})\text{d}{\bf x}+\sum_{i}\frac{1}{2}Ld_{N_{side}}S_{N_{side}}
	\]
	and 
	\[
	\left|S_{N_{side}}\sum_{i}f({\bf x}_{i})-\int f({\bf x})\text{d}{\bf x}\right|\le\frac{1}{2}Ld_{N_{side}}S_{tot}
	\]
	where $S_{tot}$ is the the total suface of the sphere. Not that inequality
	directly replace the Hoeffdings inequality in Belkin and Niogi. Now
	we need to characterize $d_{N_{side}}$ with respect of $N_{side}$
	and $L$ with respect of $m$ and $\ell$. The rest of the proof should
	be similar...
\end{itemize}

\subsection*{Non-uniform sampling}

\subsubsection*{Case 1: random}

Just use the theorem of Belkin and Niogi

\subsubsection*{Case 2: deterministic}

Use the porte-manteau theorem

\subsection*{KNN HEALPix graph}

I believe there is no convergence because we will always have irregularities
in the special points... Can we prove it? Probably... Use the fact
tha the graph is made of 12 similar parts.

