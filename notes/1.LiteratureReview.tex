
%*******************************************************************************
%*********************************** First Chapter *****************************
%*******************************************************************************
%!TEX root = 0.main.tex

\setcounter{page}{1}
\graphicspath{{Figs/LiteratureReview}}



%********************************** %First Section  **************************************
\section {Introductive Study, Literature Review} 

\subsection{The Graph Laplacian (follwing Belkin and Niyogi)}
In summary, the heat equation is indeed the key to approximating the Laplace operator. On can observe that given the heat equation 
\begin{equation}
\frac{\partial}{\partial t}u(\mathbf x, t)-\triangle u(\mathbf x, t) = 0
\end{equation}
the corresponding solution, for an initial heat distribution $f(\mathbf x)$ is given by $\mathbf{H}^t f(\mathbf x) $
where $\mathbf{H}^t f(\mathbf x)$ is the heat kernel convolution operator:
$$\mathbf{H}^t f(\mathbf x) = \int_{\mathcal R^k}f(\mathbf y)H^t(\mathbf x, \mathbf y) d\mathbf y$$
$$H^t(\mathbf x, \mathbf y)  = (4\pi t)^{-\frac{k}{2}}e^{-\frac{||\mathbf x - \mathbf y||^2}{4t}}$$

Furthermore, it can be proved that 
$$\lim_{t\rightarrow 0} \mathbf{H}^t f(\mathbf x) = f(\mathbf x) $$
So, the Laplacian of the initial distribution f can be written in the following way:
\begin{equation}
-\triangle f(\mathbf x) = -\frac{\partial}{\partial t}\mathbf{H}^t f(\mathbf x) |_{t=0}
\end{equation}

Rewriting the rightmost term we obtain

$$-\frac{\partial}{\partial t}\mathbf{H}^t f(\mathbf x) |_{t=0}  = \lim_{t\rightarrow 0} \frac{1}{t}\left( \int_{\mathbb R^k}f(\mathbf y)H^t(\mathbf x, \mathbf y)d\mathbf y - f(\mathbf x)\right)$$
And thus the Laplacian can be written as follows:
$$-\triangle f(\mathbf x) = \lim_{t\rightarrow 0} \frac{1}{t}\left((4\pi t)^{-\frac{k}{2}} \int_{\mathbb R^k}f(\mathbf y)e^{-\frac{||\mathbf x - \mathbf y||^2}{4t}}d\mathbf y - f(\mathbf x)(4\pi t)^{-\frac{k}{2}} \int_{\mathbb R^k} e^{-\frac{||\mathbf x - \mathbf y||^2}{4t}}d\mathbf y \right)$$

Writing the discrete version of the integrals involved using the point cloud $\mathcal S$ we have
\begin{equation}\label{eq:discrete_laplacian}
	\hat \triangle_\mathcal S f(\mathbf x) =  \frac{1}{t} \frac{(4\pi t)^{-\frac{k}{2}}}{n}\left( f(\mathbf x)\sum_{i} e^{-\frac{||\mathbf x_i - \mathbf x||^2}{4t}} - \sum_{i} f(\mathbf x_i) e^{-\frac{||\mathbf x_i - \mathbf p||^2}{4t}}\right)
\end{equation}

We are now ready to define the Graph Laplacian:
\begin{definition}{[\textit{Graph Laplacian}]}\\
	For a graph $G = (V, E)$ defined by the weight matrix $W$ we define the Graph Laplacian $L$
	$$L = D-W$$
	Where $D$ is a diagonal matrix such that $D(i,i) = \sum_j W(i,j)$	
	\label{def:graph_laplacian}
\end{definition}
\textbf{By defining a \textit{full graph} on the point cloud $\mathcal S$ through the special weight matrix $W_n^t  = e^{-\frac{||x_i-x_j||^2}{4t}}$ we can write the Laplacian on such particular graph as follows:}
	
$$\mathbf L_n^tf(x) = f(x)\sum_{j}e^{-\frac{||x-x_j||^2}{4t}} - \sum_jf(x_j)e^{-\frac{||x-x_j||^2}{4t}}$$
	
Putting eq. \ref{eq:discrete_laplacian} and definition \ref{def:graph_laplacian} together we arrive at the following equation, that 
$$\hat \triangle_\mathcal S f(\mathbf x) =  \frac{(4\pi t)^{-\frac{k}{2}}}{n} \mathbf L_n^t(f)(\mathbf x)$$



\subsection{Towards a Theoretical Foundation for Laplacian-Based Manifold Methods (Belkin \& Niyogi, 2005)}
In this paper it is proved that the action of the \textit{random, extended} graph Laplacian defined with the special weight matrix $W_n^t  = e^{-\frac{||x_i-x_j||^2}{4t}}$ of a random point cloud converges \textbf{in a very mild sense (from which the title "towards a ...")} to the action of the Laplace-Beltrami operator on a general(!!) manifold $\mathcal M$ \textit{in probability}.\\
In this paper they prove two results: a punctual one for each point of the data space, and it is 

Given a set of points $\mathcal S_n = \{x_1, ..., x_n\} \subset \mathbb R^k$, the weight matrix is set to be 

$$W_n^t(i,j) = e^{-\frac{||x_i-x_j||^2}{4t}}$$

This weight matrix is set to be this way in analogy with the Heat Kernel on $\mathbb R^k$.\\

The main two results presented in such paper are the following:

\begin{theorem}{[\textit{Convergence of the Random Graph Laplacian}]}
	\\Let data points $\{x_1, ..., x_n\}$ in $\mathbb R^N$ be sampled form a uniform distribution on a manifold $\mathcal M \subset \mathbb R^N$. Put $t_n = n^{-\frac{1}{k+2+\alpha}}$ , where $\alpha>0$ and let $f\in C^\infty(\mathcal M)$.\\
	Then there is a consant $C$ such that \textit{in probability}
	$$\lim_{n\rightarrow\infty} C \frac{(4\pi t_n)^{-\frac{k+2}{2}}}{n} L_n^{t_n}f(x) = \triangle_{\mathcal M}f(x)$$
\end{theorem}

\begin{theorem}{[\textit{Uniform Convergence of the Random Graph Laplacian}]}
	\\Let data points $\{x_1, ..., x_n\}$ in $\mathbb R^N$ be sampled form a uniform distribution on a compact manifold $\mathcal M \subset \mathbb R^N$. Take the space $\mathcal F = \{f\in C^\infty , \triangle f \text{is Lipschitz}\}$. Then there exists a sequence of numbers $t_n\rightarrow 0$ and a constant $C$ such that in probability

	$$\lim_{n\rightarrow\infty} \sup_{x\in \mathcal M, f\in \mathcal F} \left| C \frac{(4\pi t_n)^{-\frac{k+2}{2}}}{n} L_n^{t_n}f(x) - \triangle_{\mathcal M}f(x) \right|= 0$$
\end{theorem}

\paragraph{Differences with what we want to prove}
\begin{enumerate}
	\item Our sampling is deterministic: HealPix. Plus, the graph is not full. Deferrard \& Perraudin use the same weight scheme as Belking \& Nyiogi, connecting the $8$ (or $7^4$) neighboring pixels in the HEALpix hierarchy. However, in Belking \& Nyiogi the graph is FULL. However, as the sampling increases, maybe we can show convergence of some quantity to the integral of equation (9), and then with a plug and play the whole paper works!
	\item Our graph is not fully connected
\end{enumerate}

\paragraph{Questions}
\begin{enumerate}
	\item How come that the Laplacian defined on our graph actually seems to act as the Laplacian of Belkin \& Nyiogi? \bf{Answer}: 
\end{enumerate}
\subsection{Computing Fourier Transforms and Convolutions on the 2-Sphere (Driscoll and Healy, 1994)}
If we find a basis of minimal subspaces invariant (a vector space of functions on the sphere is invariant if all of the operators $\Lambda(g), g\in SO(3)$ take each function in the space back into the space) under all the rotations of $SO(3)$, then we simplify a lot the analysis of rotation-invariant operators.
\paragraph{Things to keep in mind from section 2, "Preliminaries"}
\begin{enumerate}
	\item any rotation $g\in SO(3)$ can be written in the well-known Euler angle decomposition: $g = u(\phi)a(\theta)u(\psi)$ determined uniquely for almost all $g$. Remember that any point on the sphere
	\item $\omega(\theta, \phi) = \left(\cos\phi\sin\theta, \sin\phi\sin\theta, \cos\theta\right)$. In fact, the 2-sphere is a quotient of the rotation group $SO(3)$ and inherits its natural coordinate system from that of the group.
	\item $\Lambda(g)f(\omega) = f(g^{-1}\omega)$
	\item invariant volume measure on $SO(3)$ is $dg=\sin\theta\ d\theta\ d\phi\ d\psi$, invariant volume measure on the sphere is $d\omega = \sin\theta\ d\theta\ d\phi$
	\item The invariant subspace of degree $l$ harmonic polynomials restricted to the sphere is called the space of \textit{spherical harmonics of degree l}. Spherical harmonics of different degree are orthogonal to one another
	\item In coordinates, 
	$$Y_l^m(\theta, \phi) =(-1)^m\sqrt{\frac{(2l+1)(l-m)!}{4\pi(l+m)!}}P_l^m(\cos\theta)e^{im\phi}$$
	where $P_l^m$ are Legendre functions.
	\item Of all the possible basis for $L^2(S^2)$ the spherical harmonics uniquely exploit the symmetries of the sphere. Under a rotation $g$, each spherical harmonic of degree $l$ is transformed into a linear combination of only those spherical harmonics of same degree $l$.
	Thus the effect of a rotation on a function expressed in the basis of the spherical harmonics is a multiplication by a semi-infinite block-diagonal matrix with the $(2l+1)\times(2l+1)$ blocks for each $l \geq 0$ given by $$D^{(l)}(g) = \left(D^{(l)}_{m,n}\right) (g) =  \left(D^{(l)}_{m,n}\right)(u(\phi)a(\theta)u(\psi)) = e^{-im\psi}d^{(l)}_{m,n}(\cos \theta) e^{-in\phi}$$
	The effect of all of this is to block-diagonalize rotationally invariant operators; namely, convolution operators obtained as weighted averages of the rotation operators by functions or kernels. For example the Laplace-Beltrami operator, that acts diagonally on the spherical harmonic basis.
	\item \begin{definition}{[\textit{Left Convolution}]}\\
		$k\star f(\omega) = \left(\int_{g\in SO(3)}dg\ k(g\eta)\Lambda(g)\right)f(\omega) = \int_{g\in SO(3)}k(g\eta)f(g^{-1}\omega)dg$
		where $\eta$ is the north pole. 
	\end{definition}
Since the convolution is a linear combination of rotation operators $\Lambda(g)$, it follows that also the convolution must be block diagonalized. Indeed,
$$\hat {(f \star h)}(l,m) = 2\pi \sqrt{\frac{4\pi}{2l+1}}\hat f(l,m) \hat h(l,0) $$

\item We desire the ability to sample a band-limited function on the sphere so that the integrals defining the Fourier coefficients can be efficiently evaluated as weighted sums of the samples. In this paper the authors present a sampling result for band-limited functions that can exactly recover the original function obtaining its transform as weighted sum of the sampled values of that function, \textbf{given an equiangular sampling of the sphere}. This is stated in the following theorem:
\begin{theorem}{[\textit{Shannon on the sphere}]}\\
	Let $f(\theta, \phi)$ be a band-limited function on $S^2$ such that $\hat f(l,m) = 0$ for $l\geq b$. Then
	\begin{equation}
			\hat f(m,l) = \frac{\sqrt{2\pi}}{2b}\sum_{j=0}^{2b-1}\sum_{k=0}^{2b-1}a_j^{(b)}f(\theta_j, \phi_k)\bar Y_l^m(\theta_j, \phi_k)
			\label{eq:shannon_on_the_sphere}
	\end{equation}

	for $l\leq b, |m|\leq l$. Here $\theta_j = \frac{\pi j}{2b}$, $\phi_k = \frac{\pi k}{b}$, and the coefficients $a_k^{(b)}$ are defined in equation (5) of that paper.
\end{theorem} 

\item To be efficient to calculate the Fourier coefficients $\hat f(m,l)$ they separate the Legendre part and the exponential part  of the spherical harming to leverage a Fast-Fourier transform rewriting eq. (\ref{eq:shannon_on_the_sphere}) in the following way
\begin{equation}
		\hat f(m,l) = q^m_l\sum_{k=0}^{b-1}a_k^{(b/2)}P_l^m(\cos\ k\theta)\sum_{j=0}^{b-1} e^{-imj\phi}f(k\theta, j\phi)
\end{equation}
where $\theta = \pi/b$, $\phi = 2\pi/b$. In this way the inner sum can be calculated for each fixed $k$ and for all $m$ by means fo the fast Fourier Transform. Once obtained the inner sum, the outer sum can be calculated with a Legendre transform.
\end{enumerate}
